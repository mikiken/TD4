\documentclass[dvipdfmx,a4j, titlepage]{jsarticle}
\usepackage{listings,jvlisting,array,tcolorbox,ascmac,siunitx,amsmath,amssymb,bm,here,float,comment,url}

% 以下の内容を記載
% 1. 表紙(名前と学籍番号含む)
% 2. 仕様書(何を作ったか.何故作ったか.どのように作ったか.)
% 3. 回路の説明(全体図は必ず入れる.部分的な図は説明に必要なものは入れる.)
% 4. 考察,まとめ
% 5. 感想,学び

% TODO: 
% - 学籍番号の挿入
% - 回路図pdfの結合

\lstset{
    basicstyle=\ttfamily, % 等幅フォントを指定
    columns=fullflexible, % 文字の間隔を狭める
    keywordstyle=\color{blue},
    commentstyle=\color{green!40!black},
    stringstyle=\color{orange},
    numbers=left,
    numberstyle=\tiny,
    stepnumber=1,
    numbersep=5pt,
    frame=single,
    rulecolor=\color{black},
    captionpos=b,
    breaklines=true,
    breakatwhitespace=false,
    tabsize=4
}

\title{電気電子工学実験3 テーマS(a)\\ 簡単な4bit CPUの製作}
\author{三木 健太郎}
\date{2023年11月20日}
 
\begin{document}

\maketitle

\section{概要}
\subsection{製作物の概要}
本テーマ後半の自由設計では、簡単な4bit CPUの製作を行った。
このCPUはデータ転送命令や加算命令、データの入出力命令などを実行することができる。
これらの命令を組み合わせることで、LEDを予め決めた通りのパターンで光らせるプログラムや、
タイマー機能を持つプログラムなどを実行することができる。

\subsection{書籍「CPUの創りかた」について}
74シリーズの汎用論理ICを用い、データ転送・加算・データ入出力といった命令を実行できる4bit CPUを製作する方法を解説した書籍である。
LEDの点灯回路といった基本的な事項から、CPUの基本構成、実装方法といった応用的な事項まで、幅広い内容が平易に解説されている。
2003年に毎日コミュニケーションズ(現 マイナビ出版)から出版され、現在30刷以上重版されている。

\subsection{製作の動機}
計算機工学1や2で、フリップフロップなどのディジタル回路の基本的な構成要素やCPUの構造について学習した。
しかし、授業で知識を学んだだけでは、CPUの内部構造について、深い理解を得ることはできなかった。
そこで、本実験において簡単なCPUを実装することにより、CPUの内部構造を実感をもって理解したいと考えた。
簡単なCPUを実装するという内容の書籍は複数存在するが、回路自体の規模が大きく、授業時間内で実装を完遂するのが難しいものが多かった。
「CPUの創りかた」で解説されている「TD4」は、機能を絞っている分、回路の規模が小さく、授業時間内で実装を完遂することが可能であると考えた。

\subsection{制作方法}
基本的には、書籍「CPUの創りかた」に従って製作を行った。
具体的には、レジスタ、ALU、プログラムカウンタ、命令デコーダ、ROMの順に実装を行った。
書籍では74シリーズのICを用いながら実際に回路を組み立てるため、
解説内容がFPGAでの実装にそぐわない部分が一部あった。
そのような部分については、FPGAでの実装に適した内容に書き換えた。(詳細は後述する。)

\section{回路の説明}
\subsection{TD4の仕様}
TD4の命令長は8bitとなっており、オペコードが4bit、オペランドが4bitとなっている。
\footnote { 表現できるアドレスも4bitの範囲となるため、実行できるプログラムは最大で16ステップのものまでとなる。 }
演算用のレジスタはAレジスタとBレジスタの2つあり、いずれも4bitの値を記憶することができる。\\
TD4において実行できる命令の一覧を、 以下の表\ref{instruction_list}に示す。

\begin{table}[hbtp]
    \caption{TD4で実行可能な命令の一覧}
    \label{instruction_list}
    \centering
    \begin{tabular}{ccc}
        \hline
        命令                                  & オペコード & 概要                                                                       \\
        \hline \hline
        \verb|ADD A, Im|                & 0000       & Aレジスタに即値\verb|Im|を加算する                            \\
        \verb|ADD B, Im|                & 0101       & Bレジスタに即値\verb|Im|を加算する                            \\
        \verb|MOV A, Im|                & 0011       & Aレジスタに即値\verb|Im|を代入する                            \\
        \verb|MOV B, Im|                & 0111       & Bレジスタに即値\verb|Im|を代入する                            \\
        \verb|MOV A, B|                & 0001       & Bレジスタの値をAレジスタに代入する                                         \\
        \verb|MOV B, A|               & 0100       & Aレジスタの値をBレジスタに代入する                                         \\
        \verb|JMP Im|               & 1111       & 即値\verb|Im|で指定されたアドレスにジャンプする              \\
        \verb|JNC Im|               & 1110       & キャリーフラグが0のとき即値\verb|Im|のアドレスにジャンプする \\
        \verb|IN A|               & 0010       & 入力端子からデータを入力し、Aレジスタに代入する                            \\
        \verb|IN B|               & 0110       & 入力端子からデータを入力し、Bレジスタに代入する                            \\
        \verb|OUT B| \footnotemark & 1001       & Bレジスタの値を出力端子に出力する                                          \\
        \verb|OUT Im|               & 1011       & 即値\verb|Im|を出力端子に出力する                            \\
        \hline
    \end{tabular}
\end{table}
\footnotetext{なお、\verb|OUT A|命令は存在しない。}

\subsection{回路の全体像}
今回製作した回路の全体像を、以下の図\ref{circuit_diagram}に示す。なお、高解像度の回路図を本レポートの末尾に付した。

\begin{figure}[H]
    \begin{center}
        \includegraphics[width=165mm]{img/circuit_diagram.jpg}
    \end{center}
    \caption{製作した回路の全体像}
    \label{circuit_diagram}
\end{figure}

図\ref{circuit_diagram}の"クロック生成"と書かれた部分では、7490を用いて評価ボード(MU500-RXSET01)の生成するクロックを、人間にも確認しやすい程度の周波数(数Hz程度)に落としている。\\

"レジスタ"は、プログラムの実行に必要な値やアドレスを格納している。上の2つのレジスタは演算用のレジスタであり、ROMから読みだした値やアドレスを格納したり、ALUでの演算結果を格納したりするのに用いられる。また、その下のレジスタは出力端子に直結しており、出力端子に出力する値を保持するために用いられている。一番下のレジスタはプログラムカウンタであり、現在CPUが実行している命令のアドレスを記憶している。\\

その右の"データセレクタ"は、命令デコーダの出力を元に、指定されたレジスタから値を読み出し、ALUへと送る役割を担っている。\\

"ALU"は、Arithmetic Logic Unit (算術論理演算装置)の略であり、算術演算や論理演算を行う回路である。今回製作したCPUでは、算術命令は\verb|ADD|命令(加算命令)しか存在しないため、ALUとして単に74283 (4bit全加算器)を用いている。\\

"命令デコーダ"は、ROMから読みだした命令の上位4bitのオペコードを見て、データセレクタがデータを読み出すべきレジスタを指定するとともに、各レジスタに対して値をロードするか保持するかを決定している。\\

"キャリーフラグ"は、D-FFを用いて\verb|ADD|命令において繰り上がりが発生した場合に、繰り上がりが発生したことを示すフラグ(キャリーフラグ)を記憶している。ここで記憶されたフラグは、\verb|JNC|命令において、オペランドとして受け取ったアドレスにジャンプするかどうかの判定に用いられる。\\

"ROM"は、Read Only Memoryの略であり、CPUで実行するプログラムを記憶している。この回路のROMは16個のアドレス空間を持ち、アドレス1つ1つに対して命令が記録されている。例えば、ROMの\verb|0010|番地に\verb|01010010| (=\verb|ADD B, 0010|)という命令が記録されていた場合、プログラムカウンタが\verb|0010|を指したときに、CPUはBレジスタに2を加算する命令を実行する。ROMの実装の詳細については、\ref{implemtnt_rom}節で述べる。\\

\subsection{CPUの動作の流れ}
CPUはリセットがかかると、ROMの\verb|0000|番地の命令から実行を開始していく。ROMに記録されている機械語命令のうち、上位4bitはオペコード(命令の種類)を表しており、下位4bitはオペランド(演算に用いる値やアドレス)を表している。オペコードは命令デコーダに読み込まれ、オペランドはALUに送られる。次に、読み込んだオペコ―ドやオペランドを元に命令を実行する。例えば、ALUで加算を行ったり、レジスタに値をロードしたりする。演算結果はレジスタに格納され、次以降の命令に利用する。最後にプログラムカウンタの値を1増加させ、次に\verb|0001|番地の命令を同様に実行する。こうした動作をクロックに合わせて繰り返し行うことで、CPUは動作する。

\subsection{ROMの実装方法}\label{implemtnt_rom}
書籍「CPUの創りかた」では、8bitのDIPスイッチを16個用意し、スイッチのOFF/ONを\verb|0/1|に対応させることで、ROMを表現していた。しかし、今回実装に使用したFPGAボードであるMU500-RXSET01
には、8bitのDIPスイッチ2個しか搭載されていなかったため、書籍の通りにROMを実装することはできなかった。\\
そこで、Verilog HDLを用いて、以下のような素子を作成することで、ROMを表現した。\\

\stepcounter{figure}
\begin{lstlisting}[language=Verilog, label={rom_code}, title={図\thefigure ROMを表すVerilogコード}]
module rom(
    input  [3:0] address,
    output [7:0] data
);
    function [7:0] select;
        input [3:0]	address;
        case (address)
            //Ramen timer!
            4'b0000: select = 8'b10110111; // OUT 0111
            4'b0001: select = 8'b00000001; // ADD A,0001
            4'b0010: select = 8'b11100001; // JNC 0001
            4'b0011: select = 8'b00000001; // ADD A,0001
            4'b0100: select = 8'b11100011; // JNC 0011
            4'b0101: select = 8'b10110110; // OUT 0110
            4'b0110: select = 8'b00000001; // ADD A,0001
            4'b0111: select = 8'b11100110; // JNC 0110
            4'b1000: select = 8'b00000001; // ADD A,0001
            4'b1001: select = 8'b11101000; // JNC 1000
            4'b1010: select = 8'b10110000; // OUT 0000
            4'b1011: select = 8'b10110100; // OUT 0100
            4'b1100: select = 8'b00000001; // ADD 0001
            4'b1101: select = 8'b11101010; // JNC 1010
            4'b1110: select = 8'b10111000; // OUT 1000
            4'b1111: select = 8'b11111111; // JMP 1111
            default: select = 8'bxxxxxxxx;
        endcase
    endfunction
    assign data = select(address);
endmodule
\end{lstlisting}

図\thefigure のように、入力として4bitのアドレスを受け取り、そのアドレスに記録されている命令を出力するような素子を作成した。Quartus Primeの、Verilogコードを回路上のシンボルに変換する機能を利用し、このVerologコードを回路内に組み込んだ。\\

\subsection{サンプルプログラム}
サンプルプログラムとして、書籍でも紹介されている「LEDを決まったパターンで点滅させるプログラム」と「タイマープログラム」を作成した。

\subsubsection{LEDを決まったパターンで点滅させるプログラム}
今回製作したTD4で実行できるプログラムとして、まず簡単な「LEDを決まったパターンで点滅させるプログラム」を作成した。実際のプログラムを以下に示す。\\

\stepcounter{figure}
\begin{lstlisting}[language=Verilog, label={led_chikachika}, title={図\thefigure LEDを決まったパターンで点滅させるプログラム}]
//LED chikachika!
4'b0000: select = 8'b10110011; // OUT 0011
4'b0001: select = 8'b10110110; // OUT 0110
4'b0010: select = 8'b10111100; // OUT 1100
4'b0011: select = 8'b10111000; // OUT 1000
4'b0100: select = 8'b10111000; // OUT 1000
4'b0101: select = 8'b10111100; // OUT 1100
4'b0110: select = 8'b10110110; // OUT 0110
4'b0111: select = 8'b10110011; // OUT 0011
4'b1000: select = 8'b10110001; // OUT 0001
4'b1001: select = 8'b11110000; // JMP 0000
4'b1010: select = 8'b00000000;
4'b1011: select = 8'b00000000;
4'b1100: select = 8'b00000000;
4'b1101: select = 8'b00000000;
4'b1110: select = 8'b00000000;
4'b1111: select = 8'b00000000;
default: select = 8'bxxxxxxxx;
\end{lstlisting}

今回製作したTD4は、出力端子として4つのLEDを備えており、OUT命令で1を指定したbitに対応するLEDが点灯する。このプログラムを実行した場合、隣り合う2つのLEDが左右に繰り返し往復するように点滅する。\\

\subsubsection{タイマープログラム}
次に、もう少し応用的なプログラムとして、「タイマープログラム」を作成した。実際のプログラムを以下に示す。\\

\stepcounter{figure}
\begin{lstlisting}[language=Verilog, label={ramen_timer}, title={図\thefigure タイマープログラム}]
//Ramen timer!
4'b0000: select = 8'b10110111; // OUT 0111
4'b0001: select = 8'b00000001; // ADD A,0001
4'b0010: select = 8'b11100001; // JNC 0001
4'b0011: select = 8'b00000001; // ADD A,0001
4'b0100: select = 8'b11100011; // JNC 0011
4'b0101: select = 8'b10110110; // OUT 0110
4'b0110: select = 8'b00000001; // ADD A,0001
4'b0111: select = 8'b11100110; // JNC 0110
4'b1000: select = 8'b00000001; // ADD A,0001
4'b1001: select = 8'b11101000; // JNC 1000
4'b1010: select = 8'b10110000; // OUT 0000
4'b1011: select = 8'b10110100; // OUT 0100
4'b1100: select = 8'b00000001; // ADD 0001
4'b1101: select = 8'b11101010; // JNC 1010
4'b1110: select = 8'b10111000; // OUT 1000
4'b1111: select = 8'b11111111; // JMP 1111
default: select = 8'bxxxxxxxx;
\end{lstlisting}

このプログラムを実行すると、まず評価ボード上のLEDが3つ点灯する。クロックを16回刻むたびに、LEDが1つずつ消灯していく。最後のLEDが消灯すると、タイマーのカウントが終了し、カウント終了を示すLEDが点灯する。\footnote{なお、出力端子にブザーを接続しておくことで、タイマーのカウント終了を音で知らせることも可能である。}\\
プログラムを確認すると、LEDが3つ点灯する部分が、図\ref{ramen_timer}の1行目の命令に対応している。その次の行でAレジスタに1を加算している。このプログラムにおけるポイントは、この次の行の\verb|JNC 0001|である。\verb|JNC|命令は直前の加算命令において、繰り上がりが発生した場合は、単に次の命令を実行し、発生しなかった場合は、オペランドで指定されたアドレスにジャンプする命令である。Aレジスタの初期値は0であるため、Aレジスタの値が\verb|10000|(=16)となる(繰り上がりが発生する)までは、1行前の\verb|ADD A,0001|にジャンプし、Aレジスタの値が増え続ける。Aレジスタの値が\verb|10000|となり、繰り上がりが発生すると、1行前の\verb|ADD A,0001|に戻ることなく、次の行の命令が実行される。このように、\verb|JNC|命令と\verb|ADD|命令を上手く組み合わせることで、上記のような動作を実現している。\\

\section{考察}
書籍の記述に従って実装を進めていったため、比較的スムーズに実装を行うことができたと感じている。しかし、\ref{implemtnt_rom}節で述べたROMの実装方法のように、書籍の記述をそのまま実装に用いることができない場合もあったため、工夫を要した。一方で、FPGAを用いて実装したことで、書籍の記述より簡単に実装を行うことができた箇所もあった。例えば、クロック生成回路は、書籍上では発振回路とシュミットトリガを利用し、クロックの波形を生成するといった解説がなされていたが、今回の実装では評価ボード上のクロックをそのまま利用するだけで済んだ。FPGAで回路を実装していると、作成している回路がアナログ回路として実際にどのように動作するのか、という視点を忘れがちであるが、トラブルシューティングのためにもこうした視点を忘れないように意識したいと感じた。(例えば、信号の立ち上がりに有限の時間を要することなど)\\

\section{感想・学び}
簡単なCPUの製作を通して、ブラックボックスだと思っていたCPUの構造や動作原理についての理解が深まったと感じる。しかし、授業時間内という制約のもとで、本の内容を一通り実装したため、内容を完全に理解しきれたとはいえない。本を再び読み直して、理解を深めたいと考えている。また、近年では個人でもFPGAボードが比較的安価に入手できるようになっているため、もう少し大規模なCPUをハードウェア記述言語を用いて実装してみたいと感じた。

\end{document}
